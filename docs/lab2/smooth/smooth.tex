%!TEX program=xelatex
%!TEX spellcheck=en_US
\documentclass[final]{report}
\input{../../.library/preamble.tex}
\input{../../.library/style.tex}
\addbibresource{../../.library/bibliography.bib}
\begin{document}
\chapter{Smoothing}

\section{Kernel Implementation}
The kernel for the smoothing algorithm is similar to the approach described in \cref{sec:rgb2gray}.
A good approach to speed up this kernel would be to use the NVIDIA NPP library.
NVIDIA NPP is a library of functions for performing CUDA accelerated processing and it has special functionality for image filters.
However, implementing these library functions is beyond the scope of this report.

\section{Results}
\subimport{resources/}{perf-tableshared.tex}

\cref{tab:smooth-results} shows a familiar sight, image09 has the highest speedup and the other images follow in descending order of image size.
Again, the speedup metrics can be explained with the kernel runtime.
For all three images the GPU kernel runtime is almost similar, however the sequential runtime differs a lot.
The longer the sequential kernel takes, the bigger impact it has on the overall running time and the bigger the speedup becomes when improved with a parallel running GPU kernel.

%\footnotetext{http://docs.nvidia.com/cuda/pdf/NPP\_Library\_Image\_Filters.pdf}

\end{document}