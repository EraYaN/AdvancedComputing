%!TEX program=xelatex
%!TEX spellcheck=en_US
\documentclass[final]{report}
\input{../../.library/preamble.tex}
\input{../../.library/style.tex}
\addbibresource{../../.library/bibliography.bib}
\begin{document}
\chapter{Initial OpenCL Implementation}\label{ch:initopencl}

\section{Introduction}
The code to start with already contained a rough outline of the OpenCL code structure.
Most parts were fairly straightforward, however, implementing the kernels proved to be more difficult, the sections below will elaborate on this.

\section{Main Function}
The framework around the two kernels is very similar to that of the CUDA implementation.
\cref{sec:code-error}, describes one of the fundamental differences between the two implementations.
Other than that the OpenCl code is much bigger, since buffer creation but also memory copying is much more verbose in OpenCL as compared to CUDA.

\section{Neighbour/Compute Kernel}
Both the neighbour and compute kernel implementations are very similar to that of the CUDA code.
\cref{lst:cuda-opencl-neighbour}, shows the main things that changed when porting the code to OpenCL.

\includecode[cpp]{OpenCL equivalent}{resources/cuda-vs-opencl-neighbour-snippet.cpp}{lst:cuda-opencl-neighbour}

\section{Fixing an Error in the Provided Code}\label{sec:code-error}
After having set up all the right variables and kernel implementations, the OpenCL implementation seemed to work.
However, after comparing its output to that of the CPU en CUDA implementation, it became clear that the axon voltage computation was wrong.
Since hardly any debug tools are available for OpenCL, the entire compute kernel was fitted with \texttt{printf} statements to monitor each computation step.
By comparing each step between the CPU and OpenCL code, the mistake was found to be in the \texttt{IcNeighbours} function.
The problem was in the index of \texttt{neighVdend}, namely, \texttt{V = prevV\_dend - neighVdend[i]\;}, which started iterating at 0.
The CPU implementation \texttt{cellCompParams} contains the neighbours and states in a struct format, it also has a separate value for \texttt{iApp}.
Therefore, one can start iterating through the neighbours at 0, however, the OpenCL implementation is one single array and its neighbours start at position 1 in the array, not 0.
The 0 spot is reserved for the \texttt{iApp} value.
Thus the code was changed to, \texttt{V = prevV\_dend - neighVdend[VNEIGHSTARTADD + i]\;}, after which the program was fully operational with the correct output.

\end{document}
