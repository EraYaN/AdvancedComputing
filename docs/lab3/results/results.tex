%!TEX program=xelatex
%!TEX spellcheck=en_US
\documentclass[final]{report}
\input{../../.library/preamble.tex}
\input{../../.library/style.tex}
\addbibresource{../../.library/bibliography.bib}
\begin{document}
\chapter{Results}

\begin{table}[H]
	\centering
	\caption{Benchmarking results, total and kernel time for CPU, CUDA and OpenCL}
	\label{tab:results}
	\begin{tabular}{ccccccc}
		\toprule
			\textbf{Stage}		& \textbf{CPU Tot.}	& \textbf{CUDA Tot.}	& \textbf{OpenCL Tot.}	& \textbf{CPU Ker.}	& \textbf{CUDA Ker.}	& \textbf{OpenCL Ker.}\\
		\midrule
			\textit{Stage 1}	& 77.1 (s)			& 56.7 (6.36)			& 18.8 (s) 				& 37.1 (s) 			& 1.92 (6.3)			& 14.5 (s)	\\
			\textit{Stage 2}	& 77.5 (s)			& 6.4 (s) 				& 19.4 (s) 				& 37.4 (s) 			& 6.4 (s) 				& 15.2 (s)	\\
			\textit{Stage 3}	& 54.3 (77.3)		& 6.32 (s) 				& 16.4 (s) 				& 40.3 (37.2) 		& 6.3 (s) 				& 12.5 (s)	\\
		\bottomrule
	\end{tabular}
\end{table}

\cref{tab:results}, lists the results for the three enhancement stages.
The timing of the CUDA and CPU code is not very reliable.
Basically, the time should be the same for each stage, since the CPU and CUDA code did not receive any edits in between the stages.
During the benchmarking several IO stalls and some other unidentified problems occurred and multiple runs were not able to filter out several runtime spikes.
Therefore, the spikes of both the CPU and CUDA are filtered out by taking the average of the other two correct values.

\begin{figure}[H]
\centering
    \includegraphics[width=\textwidth]{resources/total-time-graph.pdf}
    \caption{The total execution time for each kernel per enhancement stage}
    \label{fig:total-time-graph}
\end{figure}

\begin{figure}[H]
\centering
    \includegraphics[width=\textwidth]{resources/kernel-time-graph.pdf}
    \caption{The execution time for each kernel per enhancement stage}
    \label{fig:kernel-time-graph}
\end{figure}

\begin{figure}[H]
\centering
    \includegraphics[width=\textwidth]{resources/opencl-per-stage-graph.pdf}
    \caption{The kernel execution time per enhancement stage for OpenCL}
    \label{fig:opencl-per-stage-graph}
\end{figure}

Copying the input current, \texttt{iApp} as one array did not harvest any improvements in OpenCL.
Even though the copying of the input current can be avoided, the index still needs to be copied, which effectively has zero profit.

TODO: Images are faster for neighbour 3.3 sec to 2.6, but slower for compute. (5-10\%)
TODO: note somewhere which simulation size we used
TODO: Discuss the improvement of the merged kernel

\end{document}