%!TEX program=xelatex
%!TEX spellcheck=en_US
\documentclass[final]{report}
\input{../../.library/preamble.tex}
\input{../../.library/style.tex}
\addbibresource{../../.library/bibliography.bib}
\begin{document}
\chapter{Contrast Enhancement}

\section{Kernel Implementation}
% The kernel to convert a color image to a gray one is fairly simple and straightforward.
% It is actually an exact copy of the CPU code apart from the for loops.
% The loops have been replaced by two integers \texttt{x} and \texttt{y} indicating the thread number.
% As long as these thread numbers are within the bounds of the size of the image, the exact same code as for the CPU is executed.
% See \cref{lst:rgb2graySnippet} for the most important highlights of the kernel code.

% \includecode[cpp]{Grayscale Kernel}{resources/rgb2grayKernelSnippet.cpp}{lst:rgb2graySnippet}

% Since each value is only used once, shared memory cannot be used to improve the speed of the kernel.

\section{Results}
\subimport{resources/}{perf-table.tex}


\end{document}