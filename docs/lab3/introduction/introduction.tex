%!TEX program=xelatex
%!TEX spellcheck=en_GB
\documentclass[final]{report}
\input{../../.library/preamble.tex}
\input{../../.library/style.tex}
\addbibresource{../../.library/bibliography.bib}
\begin{document}
\chapter{Introduction}

\section{Goal}
The focus of this report is to provide the reader with an OpenCL implementation of a brain simulation algorithm.
The bare bones structure of the program is discussed as well as enhancements that have been applied to improve performance.
Furthermore, key points and bottlenecks to take into account, when programming for OpenCL, are highlighted. 
In order to draw a conclusion on the performance of the implementation in OpenCL, it is compared to that of a CUDA and CPU implementation.

\section{Method}
In order to get reliable performance measurements a total of 5 runs have been done for each configuration/stage.
A Python script was in control of scheduling and averaging each run on the benchmark host.
It also collected and processed the output of the program.
Due to the total absence of proper (supported) OpenCL tools, it is very hard to profile, apart form the rudimentary information the Nsight Visual Studio profile can give.
And the way the simulation is set up (with the thousands of kernel launches), profiling is also ridiculously slow.
For example some code that will run with \SI{80}{\percent} utilization on a GTX960 with run with only \SI{40}{\percent} utilization according to the profiler.
The former was measured using Windows performance counters provided by the driver.

\cref{ch:enhancements} will elaborate on each of the different enhancement techniques.
The results are chopped up into three enhancement stages, as listed below.
\begin{enumerate}
	\item Initial OpenCL implementation (port from CUDA)
	\item Image texture, \texttt{iApp} as an array
	\item Merged kernels
\end{enumerate}

% The values in the result tables are calculated as shown in \cref{lst:table-calc-snippet}.

% \includecode[python]{The code used to calculate the values in the result tables.}{resources/tableCalcSnippet.py}{lst:table-calc-snippet}

\section{System Overview}
All the implementations discussed in this report will be benchmarked on a Intel(R) Core(TM)2 Duo CPU E8400 @ 3.00GHz and the GPU listed in \cref{tab:gpu-specs}.

\begin{table}[H]
	\centering
	\caption{NVIDIA Tesla K40c GPU specifications}
	\label{tab:gpu-specs}
	\begin{tabular}{ll}
		\toprule
			\textbf{Item} &\textbf{Description}\\
		\midrule
			\textit{Cores} & 2880	\\
			\textit{Base clock} & 745 MHz \\
			\textit{Board interface} & PCI Express Gen3x16 \\
			\textit{Memory clock} & 3.0 GHz \\
			\textit{Memory bandwidth} & 288 GB/sec \\
			\textit{Memory interface} & 384-bit \\
			\textit{Memory size} & 12 GB \\
		\bottomrule
	\end{tabular}
\end{table}


\end{document}