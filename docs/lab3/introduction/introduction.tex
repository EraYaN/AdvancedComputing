%!TEX program=xelatex
%!TEX spellcheck=en_US
\documentclass[final]{report}
\input{../../.library/preamble.tex}
\input{../../.library/style.tex}
\addbibresource{../../.library/bibliography.bib}
\begin{document}
\chapter{Introduction}

% The focus of this report is to inform the reader about the differences in performance between certain image processing steps when being executed on either a CPU or GPU.
% The report compares the performance of each kernel separately and in some cases compares different implementations of the same kernel.
% At the end of the report the general results are discussed and some recommendations regarding the GPU kernel implementations are given.

% \section{Method}
% In order to get reliable performance measurements a total of 10 runs have been done for each kernel.
% A Python script was in control of scheduling and averaging each run on the benchmark host.
% It also collected the profiling metrics and processed the output of the program.

% The values in the result tables are calculated as shown in \cref{lst:table-calc-snippet}.

% \includecode[python]{The code used to calculate the values in the result tables.}{resources/tableCalcSnippet.py}{lst:table-calc-snippet}

% The images that the kernels will be benchmarked for are listed in \cref{tab:image-information}.

% \begin{table}[H]
% 	\centering
% 	\caption{Benchmark images information}
% 	\label{tab:image-information}
% 	\begin{tabular}{llll}
% 		\toprule
% 			\textbf{Name} &\textbf{Width (pixels)} & \textbf{Height (pixels)} & \textbf{Amount (pixels)}\\
% 		\midrule
% 			Image04 & 1200				& 1398				& 1677600 \\
% 			Image09 & 8192 				& 8192 				& 67108864 \\
% 			Image15 & 3030 				& 2360 				& 7150800\\
% 		\bottomrule
% 	\end{tabular}
% \end{table}

\end{document}