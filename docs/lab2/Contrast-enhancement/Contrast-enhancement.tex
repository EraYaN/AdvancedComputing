%!TEX program=xelatex
%!TEX spellcheck=en_US
\documentclass[final]{report}
\input{../../.library/preamble.tex}
\input{../../.library/style.tex}
\addbibresource{../../.library/bibliography.bib}
\begin{document}
\chapter{Contrast Enhancement}

\section{Kernel Implementation}
The kernel for contrast is again fairly straightforward.
A similar implementation as \cref{sec:rgb2gray} had been used.
The code does not lend itself well to shared memory, as for each memory entry only one computation is done.

\section{Results}
\begin{table}[H]
	\centering
	\caption{RBG2Grayscale benchmarking results}
	\label{tab:rgb2grayscale-results}
	\begin{tabular}{llll}
	\toprule
											& \textbf{Image04} 	& \textbf{Image09} & \textbf{Image15} \\
	\midrule
	\textit{Seq. time (ms)} 				& ~ 				& ~ 				& ~ \\
	\textit{GPU time (ms)} 					& ~ 				& ~ 				& ~ \\
	\textit{Speedup} 						& ~ 				& ~ 				& ~ \\
	\midrule
	\textit{Kernel Seq. time (ms)} 			& ~ 				& ~ 				& ~ \\
	\textit{Kernel GPU time (ms)} 			& ~ 				& ~ 				& ~ \\
	\textit{Kernel Speedup} 				& ~ 				& ~ 				& ~ \\
	\midrule
	\textit{Attained GFLOP/s} 				& ~ 				& ~ 				& ~ \\
	\textit{Theoretical GFLOP/s} 			& ~ 				& ~ 				& ~ \\
	\textit{Attained Bandwidth (GB/s)}		& ~ 				& ~ 				& ~ \\
	\textit{Theoretical Bandwidth (GB/s)}	& ~ 				& ~ 				& ~ \\
	\bottomrule
	\end{tabular}
\end{table}

%TODO SSE vs AVX test. DP and SP.
% \begin{figure}[H]
% \centering
%     \setlength\figureheight{8cm}
%     \setlength\figurewidth{\linewidth}
%     \subimport{resources/}{sse-vs-avx.pdf}
%     \caption{TODO.}
%     \label{fig:sse-vs-avx}
% \end{figure}

\end{document}